\chapter{Evaluation Design}\label{C:ed}
Evaluation was carried out to compare the performance of the proposed graph database approach with the GraphEvol approach presented in \cite{2}, in order to validate the hypothesis that the graph database-based approach can produce near-optimised solutions more efficiently. More specifically, we compared the two approaches according to the following criteria:

\begin{enumerate}
  \item Best QoS-aware service composition.
  \item The average time taken to generate the Web service composition.
\end{enumerate}

\section{Datasets and Parameters} 
The collection of datasets employed in this evaluation was WSC 2008 \cite{12} which contains service collections of varying sizes. To provide a comparison with the GraphEvol approach, we ran each task independently 30 times, for each run recording the best Web service composition, fitness value and execution time. For all tests we set the weights for an objective function to the same value as was used with the GraphEvol method, namely, a fixed weight of 0.25. The other parameters for GraphEvol were: a population of 200 candidates, a mutation probability of 0.05, and a crossover probability of 0.5. Individuals were chosen for breeding using tournament selection with a tournament size of 2 \cite{2}.\par

The test to compare the performance of the graph database approach and GraphEvol approach was conducted on a Macbook Pro Mid 2012 with a 2.9GHz dual-core Intel Core i7 processor, 8GB of 1600MHz DDR3 memory, a 1536 Mb Intel HD Graphics 4000 graphics card, a 128GB solid-state drive and the OS X EI Capitan operating system.

\section{Evaluation Results} 

\subsection{Effectiveness of the Reducing Algorithm} 
To evaluate the efficacy of the reducing graph database algorithm we applied it to all test cases contained within WSC2008 \cite{12}.  Table \ref{tb:reduce} shows the number of Web services for the original repositories and the number of Web service for the reduced graph database. For example, for WSC2008 - 02, there were 558 Web services in the original repository, but only 66 of them were related to the task input and task output. For task WSC2008 - 08, there were 8138 Web service in the original repository, but only 131 of them were related to the task input and output. Thus, the number of the Web services was greatly reduced when compared to the number in the original repositories, which greatly improved the execution time when generating the Web service composition.

\begin{table}[]
\centering
\caption{Comparison of the number of services in the original repositories and the reduced repositories}
\label{tb:reduce}
\scalebox{0.9}{

\begin{tabular}{|c|c|c|ll}
\cline{1-3}
Dataset WSC2008 & Original service repository & Reduced Service repository &  &  \\ \cline{1-3}
01              & 158                         & 61                         &  &  \\ \cline{1-3}
02              & 558                         & 66                         &  &  \\ \cline{1-3}
03              & 604                         & 105                        &  &  \\ \cline{1-3}
04              & 1041                        & 46                         &  &  \\ \cline{1-3}
05              & 1091                        & 102                        &  &  \\ \cline{1-3}
06              & 2198                        & 205                        &  &  \\ \cline{1-3}
07              & 4113                        & 195                        &  &  \\ \cline{1-3}
08              & 8119                        & 131                        &  &  \\ \cline{1-3}
\end{tabular}
}
\end{table}

\subsection{Correctness of the Web Service Composition} 
An evaluation was carried out to check the correctness of the Web service compositions. The dataset employed in this evaluation was \emph{dataset 01 of WSC 2008}. The correctness of the Web service composition was determined by checking that all the service node inputs were fulfilled in the composition by virtue of being connected to a valid output. \emph{Dataset 01} contains 158 Web services. In the interest of brevity, we only discuss two compositions (Figure \ref{fig:compEg1} and Figure \ref{fig:compEg2}) generated by our approach. All examples have been manually checked and they do indeed represent each Web service composition correctly. The following examples show the task and the service compositions produced by our approach (see Figure \ref{fig:compEg1} Web service composition example 1 and Figure \ref{fig:compEg2} Web service composition example 2).\\

Task: \emph{dataset 01 WSC2008}\par
Task inputs: \emph{[inst1926141668, inst395151449, inst1557679659]}\par
Task outputs: \emph{[inst1913443608, inst664891780]}\\\par

The matching rules of Web services are validated by checking dependencies.\par
Example 1: In Figure \ref{fig:compEg1}, showing a example of Web service composition, the input of the Web service \emph{serv283321609} is \emph{I} = \emph{{ inst722854357, inst347634243, inst1881697469, inst746203847 }}. This matches the output  \emph{O} = \emph{{inst722854357, inst2092246857, inst1326239605, inst1881697469, inst1437249127, inst1519789560 }} of Web service  \emph{serv976005395}. This exact match occurs as \emph{inst722854357} and \emph{inst1881697469} are found in both Web service \emph{serv283321609} and Web service \emph{serv976005395}, while,  according to the taxonomy tree, \emph{inst1519789560} and \emph{inst1326239605} in \emph{O} are specializations of \emph{inst347634243} in \emph{I} and \emph{inst1437249127} in \emph{O} is a specialization of \emph{inst746203847} in \emph{I}.\par

Example 2:  In Figure \ref{fig:compEg2},  the input of the web service \emph{serv699915007} is \emph{I} = \emph{{inst102675811, inst1689375842, inst1716616603}}. This matches the output \emph{O$_1$} = \emph{{inst927259823,inst608977925 }} of Web service  \emph{serv2085282617} and the output \emph{O$_2$} = \emph{{inst885068313,inst1420249694,inst1488043421 }} of Web service  \emph{serv630482774}. As you can see there is no exact match between the outputs of services \emph{serv2085282617} and \emph{serv630482774}, and inputs of service \emph{serv699915007}, while, according to the taxonomy tree, \emph{inst1488043421} in \emph{O$_2$} is a specialization of inst102675811 in \emph{I}, \emph{inst927259823} in \emph{O$_1$} is a specialization of \emph{inst1689375842} in \emph{I} and  \emph{inst885068313} in \emph{O$_2$} is a specialization of \emph{inst1716616603} in \emph{I}.\par

Thus, the number of Web services involved in both examples is the same. Since we are only evaluating the correctness of the composition, we do not know whether the Web services composition is capable of performing tasks reliably or efficiently (i.e accounting for non-functional attributes).\par

The next section will discuss our comparison of the performance of our QoS-aware approach with the GraphEvol approach.\par

\subsection{Evaluation Results for QoS-Aware Service Composition} 
Table \ref{tb:evalComp},  shows a full evaluation comparing the performance of our approach and that of GraphEvol \cite{2}. The collection of datasets employed in this evaluation was WSC 2008 \cite{12}, which contains service repositories and tasks of varying complexities. Both approaches were run on the same machine (mentioned in Section 4.1).  We ran each task independently \emph{30} times, for each run recording the best Web service composition, number of services involved, composition fitness value and composition execution time. Then our algorithm calculated the mean, standard deviation for the number of services involved, fitness values and execution times separately for each task. For each run, our approach generated \emph{50} candidates and the best solution from those candidates was chosen. For GraphEvol approach, the best candidate was generated from a population size of \emph{200}.\par

\begin{table}[h]
\centering
\caption{ Average results of the tests for QoS-Aware service composition}
\label{tb:evalComp}
\scalebox{0.66}{
\begin{tabular}{|c|c|c|c|c|c|c|}
\hline
\textbf{Dataset} & \multicolumn{3}{c|}{\textbf{Graph Database (Neo4j) Approach}} & \multicolumn{3}{c|}{\textbf{GraphEvol Approach}}  \\ \hline
\textbf{2008}    & \textbf{Number}   & \textbf{Time (ms)}   & \textbf{Fitness}   & \textbf{Number} & \textbf{Time (ms)} & \textbf{Fitness}    \\ \hline
1                & 10 \rlap{{$\pm$}}\hspace{0.4cm} 0.00             & 2197.90 \rlap{{$\pm$}}\hspace{0.4cm}329 \rlap{{$\downarrow$}}          & 0.521 \rlap{{$\pm$}}\hspace{0.4cm}0.169       & 10.63 \rlap{{$\pm$}}\hspace{0.4cm}2.17      & 4845.57 \rlap{{$\pm$}}\hspace{0.4cm}315.42     & 0.645 \rlap{{$\pm$}}\hspace{0.4cm}0.139                \\ \hline
2                & 5  \rlap{{$\pm$}}\hspace{0.4cm} 0.00               & 5347.13 \rlap{{$\pm$}}\hspace{0.4cm}880           & 0.48 \rlap{{$\pm$}}\hspace{0.4cm}0.152         & 5.87 \rlap{{$\pm$}}\hspace{0.4cm}2.12       & 3699.77 \rlap{{$\pm$}}\hspace{0.4cm}364.57     & 0.906 \rlap{{$\pm$}}\hspace{0.4cm}0.00                  \\ \hline
3                & 40 \rlap{{$\pm$}}\hspace{0.4cm} 0.00               & 10961.53 \rlap{{$\pm$}}\hspace{0.4cm}790 \rlap{{$\downarrow$}}         & 0.387 \rlap{{$\pm$}}\hspace{0.4cm}0.1 \rlap{{$\uparrow$}}         & 41.2 \rlap{{$\pm$}}\hspace{0.4cm}0.79       & 17221.53 \rlap{{$\pm$}}\hspace{0.4cm}764.85    & 0.176 \rlap{{$\pm$}}\hspace{0.4cm}0.045           \\ \hline
4                & 10 \rlap{{$\pm$}}\hspace{0.4cm} 0.00              & 3885.70 \rlap{{$\pm$}}\hspace{0.4cm}399 \rlap{{$\downarrow$}}          & 0.431 \rlap{{$\pm$}}\hspace{0.4cm}0.066 \rlap{{$\uparrow$}}       & 10.2 \rlap{{$\pm$}}\hspace{0.4cm}0.48       & 6076.7 \rlap{{$\pm$}}\hspace{0.4cm}281.58      & 0.305 \rlap{{$\pm$}}\hspace{0.4cm}0.066            \\ \hline
5                & 20 \rlap{{$\pm$}}\hspace{0.4cm} 0.00             & 4510.6 \rlap{{$\pm$}}\hspace{0.4cm}468 \rlap{{$\downarrow$}}           & 0.403 \rlap{{$\pm$}}\hspace{0.4cm}0.128\rlap{{$\uparrow$}}        & 22.13 \rlap{{$\pm$}}\hspace{0.4cm}2.59      & 10444.2 \rlap{{$\pm$}}\hspace{0.4cm}572.59     & 0.164 \rlap{{$\pm$}}\hspace{0.4cm}0.046            \\ \hline
6                & 40 \rlap{{$\pm$}}\hspace{0.4cm} 0.00              & 258503.33 \rlap{{$\pm$}}\hspace{0.4cm}42324      & 0.407 \rlap{{$\pm$}}\hspace{0.4cm}0.089 \rlap{{$\uparrow$}}       & 40.2 \rlap{{$\pm$}}\hspace{0.4cm}0.48       & 22183.53 \rlap{{$\pm$}}\hspace{0.4cm}1639      & 0.228 \rlap{{$\pm$}}\hspace{0.4cm}0.06           \\ \hline
7                & 20 \rlap{{$\pm$}}\hspace{0.4cm} 0.00              & 17839.77 \rlap{{$\pm$}}\hspace{0.4cm}763 \rlap{{$\downarrow$}}          & 0.457 \rlap{{$\pm$}}\hspace{0.4cm}0.097 \rlap{{$\uparrow$}}       & 23.13 \rlap{{$\pm$}}\hspace{0.4cm}7.33      & 20304.37 \rlap{{$\pm$}}\hspace{0.4cm}1257      & 0.316 \rlap{{$\pm$}}\hspace{0.4cm}0.039          \\ \hline
8                & 30 \rlap{{$\pm$}}\hspace{0.4cm} 0.00              & 53003.7 \rlap{{$\pm$}}\hspace{0.4cm}4465         & 0.468 \rlap{{$\pm$}}\hspace{0.4cm}0.091 \rlap{{$\uparrow$}}       & 32.47 \rlap{{$\pm$}}\hspace{0.4cm}3.86      & 18567.03 \rlap{{$\pm$}}\hspace{0.4cm}2055      & 0.315 \rlap{{$\pm$}}\hspace{0.4cm}0.028                \\ \hline
\end{tabular}
}
\end{table}

% \begin{table}[h]
% \centering
% \caption{ Average results of the tests for QoS-Aware service composition}
% \label{tb:evalComp}
% \scalebox{0.66}{
% \begin{tabular}{|c|c|c|c|c|c|c|c|c|}
% \hline
% \textbf{Dataset} & \multicolumn{3}{c|}{\textbf{Graph Database (Neo4j) Approach}} & \multicolumn{3}{c|}{\textbf{GraphEvol Approach}}        & \multicolumn{2}{c|}{\textbf{T-Test P value}} \\ \hline
% \textbf{2008}    & \textbf{Number}   & \textbf{Time (ms)}   & \textbf{Fitness}   & \textbf{Number} & \textbf{Time (ms)} & \textbf{Fitness} & \textbf{Fitness}       & \textbf{Time}       \\ \hline
% 1                & 10 \rlap{{$\pm$}}\hspace{0.4cm} 0.00             & 2197.90 \rlap{{$\pm$}}\hspace{0.4cm}329 \rlap{{$\downarrow$}}          & 0.521 \rlap{{$\pm$}}\hspace{0.4cm}0.169       & 10.63 \rlap{{$\pm$}}\hspace{0.4cm}2.17      & 4845.57 \rlap{{$\pm$}}\hspace{0.4cm}315.42     & 0.645 \rlap{{$\pm$}}\hspace{0.4cm}0.139      & 0.999                  & 6.96E-38            \\ \hline
% 2                & 5  \rlap{{$\pm$}}\hspace{0.4cm} 0.00               & 5347.13 \rlap{{$\pm$}}\hspace{0.4cm}880           & 0.48 \rlap{{$\pm$}}\hspace{0.4cm}0.152         & 5.87 \rlap{{$\pm$}}\hspace{0.4cm}2.12       & 3699.77 \rlap{{$\pm$}}\hspace{0.4cm}364.57     & 0.906 \rlap{{$\pm$}}\hspace{0.4cm}0.00       & 0.9999                 & 0.99999             \\ \hline
% 3                & 40 \rlap{{$\pm$}}\hspace{0.4cm} 0.00               & 10961.53 \rlap{{$\pm$}}\hspace{0.4cm}790 \rlap{{$\downarrow$}}         & 0.387 \rlap{{$\pm$}}\hspace{0.4cm}0.1 \rlap{{$\uparrow$}}         & 41.2 \rlap{{$\pm$}}\hspace{0.4cm}0.79       & 17221.53 \rlap{{$\pm$}}\hspace{0.4cm}764.85    & 0.176 \rlap{{$\pm$}}\hspace{0.4cm}0.045      & 2.38E-13               & 2.06E-37            \\ \hline
% 4                & 10 \rlap{{$\pm$}}\hspace{0.4cm} 0.00              & 3885.70 \rlap{{$\pm$}}\hspace{0.4cm}399 \rlap{{$\downarrow$}}          & 0.431 \rlap{{$\pm$}}\hspace{0.4cm}0.066 \rlap{{$\uparrow$}}       & 10.2 \rlap{{$\pm$}}\hspace{0.4cm}0.48       & 6076.7 \rlap{{$\pm$}}\hspace{0.4cm}281.58      & 0.305 \rlap{{$\pm$}}\hspace{0.4cm}0.066      & 5.58E-110              & 3.03E-30            \\ \hline
% 5                & 20 \rlap{{$\pm$}}\hspace{0.4cm} 0.00             & 4510.6 \rlap{{$\pm$}}\hspace{0.4cm}468 \rlap{{$\downarrow$}}           & 0.403 \rlap{{$\pm$}}\hspace{0.4cm}0.128\rlap{{$\uparrow$}}        & 22.13 \rlap{{$\pm$}}\hspace{0.4cm}2.59      & 10444.2 \rlap{{$\pm$}}\hspace{0.4cm}572.59     & 0.164 \rlap{{$\pm$}}\hspace{0.4cm}0.046      & 8.80E-12               & 1.95E-44            \\ \hline
% 6                & 40 \rlap{{$\pm$}}\hspace{0.4cm} 0.00              & 258503.33 \rlap{{$\pm$}}\hspace{0.4cm}42324      & 0.407 \rlap{{$\pm$}}\hspace{0.4cm}0.089 \rlap{{$\uparrow$}}       & 40.2 \rlap{{$\pm$}}\hspace{0.4cm}0.48       & 22183.53 \rlap{{$\pm$}}\hspace{0.4cm}1639      & 0.228 \rlap{{$\pm$}}\hspace{0.4cm}0.06       & 1.81E-12               & 1.0                   \\ \hline
% 7                & 20 \rlap{{$\pm$}}\hspace{0.4cm} 0.00              & 17839.77 \rlap{{$\pm$}}\hspace{0.4cm}763 \rlap{{$\downarrow$}}          & 0.457 \rlap{{$\pm$}}\hspace{0.4cm}0.097 \rlap{{$\uparrow$}}       & 23.13 \rlap{{$\pm$}}\hspace{0.4cm}7.33      & 20304.37 \rlap{{$\pm$}}\hspace{0.4cm}1257      & 0.316 \rlap{{$\pm$}}\hspace{0.4cm}0.039      & 3.83E-09               & 3.98E-12            \\ \hline
% 8                & 30 \rlap{{$\pm$}}\hspace{0.4cm} 0.00              & 53003.7 \rlap{{$\pm$}}\hspace{0.4cm}4465         & 0.468 \rlap{{$\pm$}}\hspace{0.4cm}0.091 \rlap{{$\uparrow$}}       & 32.47 \rlap{{$\pm$}}\hspace{0.4cm}3.86      & 18567.03 \rlap{{$\pm$}}\hspace{0.4cm}2055      & 0.315 \rlap{{$\pm$}}\hspace{0.4cm}0.028      & 1.44E-10               & 1.0                   \\ \hline
% \end{tabular}
% }
% \end{table}

\subsubsection{The number of the Web services involved} 

The results in Table \ref{tb:servSize} compare the number of Web services required for a composition using two approaches. The result (number of services invoked) with the {'}\emph{minimum number of Web services}{'} approach was equal to or very close to the result using the {'}\emph{non-minimum number of Web services}{'} approach. However, the fitness values were much higher with the {'}\emph{minimum number of Web services}{'} approach. A poor result occurred with datasets \emph{01} and \emph{03}, and with dataset \emph{03} the {'}\emph{non-minimum number of Web services}{'} approach took \emph{1138667.65} milliseconds (\emph{19} minutes) to generate a single service composition due to the huge search space involved, compared with \emph{10916.53} milliseconds (\emph{10.9} seconds) using the the {'}\emph{minimum number of Web services}{'} method.  This was because when we restricted our approach to a minimum number of Web services, the search space was also restricted. The result was that when we added web service nodes into the service composition, whenever the number of the web services in the composition was greater than the minimum number of the Web services, our approach led to continual re-starting of the composition generation step in order to find a new composition.\par

So we decided to use the {'}\emph{minimum number of Web services}{'} method to generate Web service compositions in order to find the best candidates. The result was that the number of the services involved in our approach was less than or equal to the numbers of services when using the GraphEvol approach.\par

\subsubsection{Execution time to generate best QoS-Aware Web service composition} 
Our approach successfully generated a Web service composition for each task in eight datasets from WSC 2008. For five tasks our approach was significantly better than when using the GraphEvol approach in terms of execution time.\par

However, the composition generation time for the remaining three tasks was slower than when using the GraphEvol approach. This was especially true for dataset \emph{06}, which took \emph{258503.33} milliseconds using our approach compared with \emph{22183.53} milliseconds using the GraphEvol approach.\par

The poor result we obtained when using our algorithm with dataset \emph{06} occurred because dataset \emph{06} contained \emph{40} Web service nodes for the composition and the reduced graph database contained \emph{205} Web service nodes. So one of five nodes was involved in the composition leading to a large number of edges (relationships) between the Web service nodes, and the large number of service nodes and edges increased the search space, thus increasing the total execution time in generating a composition. But overall, our approach performed better than the GraphEvol approach and led to faster generation of web compositions.\par



\begin{table}[h]
\centering
\caption{ Graph Database: Average (30 independent runs) results}
\label{tb:servSize}
\scalebox{0.8}{
\begin{tabular}{|c|c|c|c|c|c|c|}
\hline
Dataset & \multicolumn{3}{c|}{Using minimum number of services} & \multicolumn{3}{c|}{Using non-minimum number of  services} \\ \hline
2008    & Number        & Time (ms)          & Fitness          & Number            & Time (ms)           & Fitness          \\ \hline
1       & 10 \rlap{{$\pm$}}\hspace{0.4cm} 0.00            & 2197.90 \rlap{{$\pm$}}\hspace{0.4cm} 329         & 0.52\rlap{{$\pm$}}\hspace{0.4cm}0.166        & 10 \rlap{{$\pm$}}\hspace{0.4cm} 0.00               & 1047.06 \rlap{{$\pm$}}\hspace{0.4cm} 367          & 0.53 \rlap{{$\pm$}}\hspace{0.4cm} 0.174        \\ \hline
2       & 5 \rlap{{$\pm$}}\hspace{0.4cm} 0.00           & 5347.13\rlap{{$\pm$}}\hspace{0.4cm}880       & 0.54\rlap{{$\pm$}}\hspace{0.4cm}0.103        & 5 \rlap{{$\pm$}}\hspace{0.4cm} 0.00                & 3699.77\rlap{{$\pm$}}\hspace{0.4cm}364.57       & 0.546\rlap{{$\pm$}}\hspace{0.4cm}0.179       \\ \hline
4       & 10 \rlap{{$\pm$}}\hspace{0.4cm} 0.00           & 3885.70\rlap{{$\pm$}}\hspace{0.4cm}399        & 0.388\rlap{{$\pm$}}\hspace{0.4cm}0.08       & 10 \rlap{{$\pm$}}\hspace{0.4cm} 0.00               & 6076.7\rlap{{$\pm$}}\hspace{0.4cm}281.58        & 0.26\rlap{{$\pm$}}\hspace{0.4cm}0.028        \\ \hline
3       & 40 \rlap{{$\pm$}}\hspace{0.4cm} 0.00           & 10961.53\rlap{{$\pm$}}\hspace{0.4cm}790         & 0.387\rlap{{$\pm$}}\hspace{0.4cm}0.1        & 40 \rlap{{$\pm$}}\hspace{0.4cm} 0.00                 & 1138667.65 \rlap{{$\pm$}}\hspace{0.4cm}83582.58                 & 0.211\rlap{{$\pm$}}\hspace{0.4cm}0.068               \\ \hline
5       & 20 \rlap{{$\pm$}}\hspace{0.4cm} 0.00           & 4510.6\rlap{{$\pm$}}\hspace{0.4cm}468         & 0.398\rlap{{$\pm$}}\hspace{0.4cm}0.121      & 20 \rlap{{$\pm$}}\hspace{0.4cm} 0.00               & 10444.2\rlap{{$\pm$}}\hspace{0.4cm}572.59       & 0.227\rlap{{$\pm$}}\hspace{0.4cm}0.065       \\ \hline
6       & 40 \rlap{{$\pm$}}\hspace{0.4cm} 0.00           & 258503.33\rlap{{$\pm$}}\hspace{0.4cm}42324       & 0.354\rlap{{$\pm$}}\hspace{0.4cm}0.103       & 42.067\rlap{{$\pm$}}\hspace{0.4cm}1.29        & 22183.53\rlap{{$\pm$}}\hspace{0.4cm}1639         & 0.206\rlap{{$\pm$}}\hspace{0.4cm}0.086       \\ \hline
7       & 20 \rlap{{$\pm$}}\hspace{0.4cm} 0.00           & 17839.77\rlap{{$\pm$}}\hspace{0.4cm}763       & 0.411\rlap{{$\pm$}}\hspace{0.4cm}0.076      & 20 \rlap{{$\pm$}}\hspace{0.4cm} 0.00               & 20304.37\rlap{{$\pm$}}\hspace{0.4cm}1257         & 0.317\rlap{{$\pm$}}\hspace{0.4cm}0.085       \\ \hline
8       & 30 \rlap{{$\pm$}}\hspace{0.4cm} 0.00           & 53003.7\rlap{{$\pm$}}\hspace{0.4cm}4465       & 0.438\rlap{{$\pm$}}\hspace{0.4cm}0.012      & 33.67\rlap{{$\pm$}}\hspace{0.4cm}2.29         & 284831.37\rlap{{$\pm$}}\hspace{0.4cm}2125         & 0.358\rlap{{$\pm$}}\hspace{0.4cm}0.052       \\ \hline
\end{tabular}
}
\end{table}

\subsubsection{Hypothesis test using two fitness value sets for each approach} 

We used a two-sample t-test to check if the fitness values and execution times for our approach were better than the fitness values and execution times for the GraphEvol approach at the \emph{0.05} significance level. In Table \ref{tb:evalComp}, it is clear that the P-values from datasets \emph{03} to \emph{08} are smaller than the significance level \emph{0.05} for the fitness values set and datasets \emph{01, 03, 04, 05} and \emph{07} are less than the significance level \emph{0.05} for the execution times set. This means there is considerable evidence that the sets of fitness values and execution times for GraphEvol approach are lower than for our approach. In other words, the best solutions in our approach are significantly superior to the best solutions produced by the GraphEvol approach. Only with datasets \emph{01} and \emph{02} did we obtain poorer fitness value sets, and only with datasets \emph{02, 06} and \emph{08} were our execution time sets poorer than with the GraphEvol approach. In summary however, our approach successfully produced a better result overall, than the GraphEvol approach.\par

\subsection{Summary}
This chapter deals with our design of an evaluation method to compare the performance of our proposed graph database-based approach with an existing GP-based approach known as GraphEvol, which is presented in \cite{2}. From our evaluation results, we concluded that our graph database-based approach leads to better performance in both finding best QoS-aware service compositions as well as the execution time taken to generate Web service compositions.\par

