\chapter{Conclusion and Future work}\label{C:conc}

This project proposed graph database-based approach to QoS-aware service composition. To do that we have proposed a model of service repository, so that information of services and relationships between services can be stored in graph databases. We have also proposed a QoS-aware service composition approach based on graph database.\par

We have tested this approach with all the service composition tasks from several collections of common benchmark test cases. We manually checked the correctness of all the inputs of service nodes in the web service composition for all the datasets in the WSC2008. Our composition correctness evaluation shows our approach is able to produce correct representations of service compositions.\par

We then extended our approach to QoS-aware web service compositions by considering additional QoS requirements. In order to compare the performance of our approach with the GraphEvol approach we used the following four QoS properties: availability, reliability, execution cost and execution duration. These QoS properties were then combined into a single fitness value for each composition. Finding the best web service composition was then possible using the fitness values of individual candidate web compositions we generated.\par

The results of our evaluation of this approach show that our algorithm can compute optimised solutions, and provides good performance comparing with an existing approach. With our approach, Web service nodes and edges (relationships) are stored in the Graph database permanently, and do not need to be re-created for each task. On the other hand, the problem of GraphEvol approach that only stores this information in memory rather than in permanent storage, and must constantly re-create relationships. This requires additional resources and leads to a longer processing time. GraphEvol also presents the additional challenge of maintaining the web service repository and the associated taxonomy tree, because of the overly complicated format used for storing data structures.\par

In this report, our approach used a reduced graph database that shrinks the search space for given task, thus improving its performance. In the future, we would like to investigate the generation of a QoS-aware web service composition using the original database, which does not decrease the performance of the solution. \par

Our approach did not save the task-related services for reuse in the future. Our future work will focus on saving all the task related services in local storage. With each task, if it is similar to a task already saved in local storage, we would reuse those saved task-related services instead of generating a task related reduced graph database from our original graph database. \par

We modelled only the service repository when developing our approach, which requires loading taxonomy.xml file from local storage. This method means it is impossible to modify the taxonomy file because of the format of the file. Moreover, whenever the taxonomy file changes, Web service graph database also needs to be updated for consistency.  Our future work on solving this shortcoming would focus on creating a graph database for taxonomy repository and trying to link all the taxonomy nodes with services' inputs and outputs. Thus every time taxonomy nodes were modified, service nodes would be automatically updated since they were interconnected. \par
