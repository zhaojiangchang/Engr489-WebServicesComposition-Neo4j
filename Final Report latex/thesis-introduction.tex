\chapter{Introduction}\label{C:intro}
Service-Oriented Architecture (SOA) \cite{1} is an architectural style for building software applications which uses services available in a network. SOA is realised through a standards-based technology called Web services, which allows coupling between Web services, so they can be reused. A Web service is a self-contained unit with limited functionality, that takes input data and produces output data \cite{27}. To provide value-added functions, it is necessary to compose Web services to provide powerful service functions. The result of such composition is to take a set of input data provided by the user and create a corresponding set of output data needed by a user.  

Currently there are three main approaches to Web service Composition. The first group of approaches includes several those which employ traditional methods such as Integer Linear Programming (ILP) \cite{7} to solve the problem of Web service composition. These approaches are lack of scalability and are no longer efficient since the number of Web services is increasing extraordinarily quickly. The second group includes various Evolutionary Computing (EC) approaches, such as Genetic Algorithms (GA) \cite{8}, Genetic Programming(GP) \cite{14,2,9} and Particle Swarm Optimisation (PSO) \cite{10,19}. These approaches are slow, especially when checking the dependencies between the component services, a task which requires a substantial amount of resources. The third type is the graph based approach, such as \cite{13,5}. This approach stores graph dependencies in memory temporarily rather than saving them permanently on local storage.


\section{Aims and Objectives}
The aim of this project is to propose an efficient automated QoS-aware Web service composition approach using Graph Databases. Existing Web service composition approaches \cite{2, 4} do not permanently store Web service dependencies, which means that when running the composition algorithm for different tasks, the algorithm will keep Web service dependencies in memory. The problem with this is that when a new task needs to be carried out, it becomes necessary to regenerate its Web service dependencies. This is because different tasks have different enquire (input) and result (outputs) data, upon which Web service dependencies are based. In practice, this behaviour dramatically increases the cost of running the system, since Web service dependencies which need only be generated once may in fact be generated several times. Moreover, the existing approach requires the generation of a corresponding workflow after each composition has been established. This takes a substantial amount of time and slows down the entire process.


Graph databases store data and  \cite{6} between data as graphs. In this project we employ a graph database to store the dependencies between services in a service repository and our service composition approach is based on the use of these graph databases. This project aims to create a graph database-based approach that generates Web service solutions efficiently by reducing the composition costs involved in checking dependencies of services. 

To achieve the aim of the project, the project is conducted with the following objectives:
\begin{enumerate}
  \item To review existing works in literature on this topic.
  \item To use a graph database to model and store services and their dependencies within a service repository.
  \item To generate non-QoS-aware Web services compositions with no redundant services in the compositions.
  \item To select service compositions with the best QoS. 
  \item To conduct a full evaluation of our approach by comparing the performance of our proposed approach with one of the existing approaches.
\end{enumerate}



\section{Structure of Report} 
This remainder of the report is organised as follows: A background and literature review is presented in Chapter 2. Chapter 3 gives a description of our Graph database-based approach. Chapter 4 covers our evaluation to compare the performance of our approach with an existing approach.  The final section of this report presents our conclusions regarding our approach, and proposes possible future developments regarding our approach.


